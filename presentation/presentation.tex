\documentclass[14pt, handout]{beamer}

% Presento style file
\usepackage{config/presento}

% Remove "figure" label in figures
\setbeamertemplate{caption}{\raggedright\insertcaption\par}

% custom command and packages
\input{config/custom-command}

\newcommand\blfootnote[1]{%
    \begingroup
    \renewcommand\thefootnote{}\footnote{#1}%
    \addtocounter{footnote}{-1}%
    \endgroup
}

% Information
\title{Extending ARGoS and TAM with Python}
\author{Alberto Parravicini}
\institute{Université libre de Bruxelles, PROJ-H-402}
\date{June 1, 2017}

\setbeamertemplate{bibliography item}[text] % No icons in bibliography

\begin{document}

% Title page
\begin{frame}[plain]
\maketitle
\end{frame}


% DEFINITIONS %%%%%%%%%%%%

\begin{frame}{A brief introduction...}  
    \begin{fullpageitemize}	
        \item<1->\textbf{ARGoS}: a simulator for swarm robotics.
        \vspace{5mm}
        \item<2->It can be extended with plugins and modules.
        \vspace{5mm}    
        \item<3->Here, 2 new extensions: 
        \begin{baseitemize}
            \item<4->[\rtarrow]A \textbf{Python wrapper}.
            \item<5->[\rtarrow]A Python implementation of \textbf{TAM} for ARGoS.
        \end{baseitemize}	
        \textbf{Goal:} simplify the user's work.
    \end{fullpageitemize}
\end{frame}

\begin{frame}{Simulations in ARGoS}  
    \begin{fullpageitemize}	
    \item<3->A simulation has 3 components:
    \begin{baseitemize}
        \item<4->[\rtarrow]A \textbf{configuration} file: to specify the entities. 
        \item<5->[\rtarrow]A \textbf{loop function}: customize events at the start/end of each step.
        \item<6->[\rtarrow]\textbf{Controllers}: program the behaviour of robots.
    \end{baseitemize}	
    \begin{figure}[H]
        \centering
        \includegraphics[width=0.7\textwidth, keepaspectratio=1]{{"images/argos_inputs_2"}.png}
        \caption{\tiny{Image from \href{http://www.argos-sim.info/user_manual.php}{http://www.argos-sim.info/user\_manual.php}}}
    \end{figure}
    \end{fullpageitemize}
\end{frame}

\begin{frame}{Python wrapper for ARGoS}
    \begin{fullpageitemize}
    \item<1->To program a controller, there are 2 choices:
    \begin{baseitemize}
        \item<2->[\rtarrow]\textbf{C++}: powerful, fast, but hard to use.
        \item<2->[\rtarrow]\textbf{Lua}: simple, but not well known language.
    \end{baseitemize}	
    \vspace{5mm}
    \item<3->A third choice:
    \begin{baseitemize}
        \item<3->[\rtarrow]\textbf{Python}: simple, and very common!
    \end{baseitemize}    
    \end{fullpageitemize}
\end{frame}

\begin{frame}{Python wrapper for ARGoS}
    \begin{fullpageitemize}
    \item<1->\textbf{Main idea:}	write a Python wrapper taking inspiration from Lua.
    \vspace{2mm}
    \begin{baseitemize}
    \item<2->[\rtarrow]Written using \textbf{Boost.Python}
    \vspace{2mm}
    \item<2->[\rtarrow]Full support for generic actuators \& sensors, \textbf{Footbot} and \textbf{e-puck}
    \vspace{2mm}
    \item<2->[\rtarrow]Same syntax as the Lua wrapper
\end{baseitemize}	

\end{fullpageitemize}
\end{frame}
\begin{frame}{PyTAM}
    \begin{fullpageitemize}
        \item<1->\textbf{TAM} is a device used for task abstractions.
        \item<2->Originally, \textbf{Java} + \textbf{C++} interface to ARGoS.
        \item<3->Now partially rewritten in \textbf{Python}, with a modified \textbf{C++} interface.
        \item<4->\textbf{PRO}: simpler code-base, and controllers writable in Python.
    \end{fullpageitemize}
\end{frame}    

\framecard[colorgreen]{{\color{white}\hugetext{THANK YOU!}}}



\end{document}